\chapter{Hipóteses dos objetivos de medição}

Essa seção de Hipóteses dos objetivos de medição é analoda à etapa de Definir/Rever Questões/Hipóteses da fase de Definição do GQM.

A partir do Tema 1, derivamos as questões referentes a esse tema:

	\begin{itemize}  
	\item Q1.1 - Qual a satisfação do cliente em relação a participação dele no processo?
	\item Q1.2 - Qual a satisfação do cliente com o produto?
	\item Q1.3 - Qual o entendimento do cliente em relação ao processo?
	\end{itemize}

A partir do Tema 2, derivamos as questões referentes a esse tema:

	\begin{itemize}  
	\item Q2.1 - Quantas atividades do modelo de maturidade aderiram ao processo?
	\item Q2.2 - Quantas atividades do modelo de maturidade foram estudadas?
	\end{itemize}

A partir do Tema 3, derivamos as questões referentes a esse tema:

	\begin{itemize}  
	\item Q3.1 - Qual o tamanho do processo atual?
	\item Q3.2 - Qual o esforço da equipe em horas?
	\item Q3.3 - Qual a produtividade da equipe?
	\item Q3.4 - Qual o tempo gasto para realização de cada atividade?
	\end{itemize}

A partir do Tema 4, derivamos as questões referentes a esse tema:

	\begin{itemize}  
	\item Q4.1 - Quanto custa o processo?
	\item Q4.2 - Quanto custa cada atividade do processo?
	\item Q4.3 - Quanto custará a aderência de uma atividade do modelo de maturidade?
	\end{itemize}