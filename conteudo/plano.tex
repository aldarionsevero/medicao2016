\chapter{Plano de Medição}

\section{Objetivos estratégicos}

Ao coletar, analizar, e atuar em cima dessas métricas aqui definidas, a equipe de Engenharia de Requisitos, juntamente com nossa equipe de Medição e Análise temos o objetivo de alcansar melhor qualidade de software para o cliente e melhor satisfação do mesmo.

\section{Medições}

\subsection{Tamanho}

	\begin{table}[H]
	\centering
	\caption{Tamanho do Processo}
	\label{my-label}
	\begin{tabular}{|l|l|}
	\hline
	Objetivo da medição & Identificar o tamanho do processo                                                                                                                                                \\ \hline
	Formula             & $\sum$(Atividades do processo)                                                                                                                                                                                 \\ \hline
	Escala da medição   & Absoluta                                                                                                                                                                         \\ \hline
	Coleta              & \begin{tabular}[c]{@{}l@{}}Responsável: Gustavo Sabino\\ Peridiocidade: Semanalmente\\ Procedimentos: Entrevista\end{tabular}                                                    \\ \hline
	Análise             & \begin{tabular}[c]{@{}l@{}}Responsável: Rafael Akiyoshi\\ Procedimentos: Nenhum. \\ Trata-se de uma medida básica\\  e será usada na análise de outras \\ mediçoes.\end{tabular} \\ \hline
	Meta                & \begin{tabular}[c]{@{}l@{}}Limites de Especificação:Inferior: 100\%\\ Superior: NA\\ Limites de Controle:Inferior: 100\%\\ Superior: NA\end{tabular}                             \\ \hline
	\end{tabular}
	\end{table}


	\begin{table}[H]
	\centering
	\caption{Tamanho da equipe}
	\label{my-label}
	\begin{tabular}{|l|l|}
	\hline
	Objetivo da medição & Identificar o tamanho da equipe                                                                                                                                                  \\ \hline
	Formula             & $\sum$(membros da equipe)                                                                                                                                                        \\ \hline
	Escala da medição   & Absoluta                                                                                                                                                                         \\ \hline
	Coleta              & \begin{tabular}[c]{@{}l@{}}Responsável: Gustavo Sabino\\ Peridiocidade: Semanalmente\\ Procedimentos: Entrevista\end{tabular}                                                    \\ \hline
	Análise             & \begin{tabular}[c]{@{}l@{}}Responsável: Rafael Akiyoshi\\ Procedimentos: Nenhum. \\ Trata-se de uma medida básica\\  e será usada na análise de outras \\ mediçoes.\end{tabular} \\ \hline
	Meta                & \begin{tabular}[c]{@{}l@{}}Limites de Especificação:Inferior: 100\%\\ Superior: NA\\ Limites de Controle:Inferior: 100\%\\ Superior: NA\end{tabular}                             \\ \hline
	\end{tabular}
	\end{table}



	\begin{table}[H]
	\centering
	\caption{Tamanho do Modelo de Maturidade}
	\label{my-label}
	\begin{tabular}{|l|l|}
	\hline
	Objetivo da medição & \begin{tabular}[c]{@{}l@{}}Identificar o tamanho do modelo \\ de maturidade no processo\end{tabular}                                                                             \\ \hline
	Formula             & \begin{tabular}[c]{@{}l@{}}$\sum$(Atividades do CMMI ou\\ MPS-BR)\end{tabular}                                                                                                   \\ \hline
	Escala da medição   & Absoluta                                                                                                                                                                         \\ \hline
	Coleta              & \begin{tabular}[c]{@{}l@{}}Responsável: Gustavo Sabino\\ Peridiocidade: Semanalmente\\ Procedimentos: Entrevista\end{tabular}                                                    \\ \hline
	Análise             & \begin{tabular}[c]{@{}l@{}}Responsável: Rafael Akiyoshi\\ Procedimentos: Nenhum. \\ Trata-se de uma medida básica\\  e será usada na análise de outras \\ mediçoes.\end{tabular} \\ \hline
	Meta                & \begin{tabular}[c]{@{}l@{}}Limites de Especificação:Inferior: 100\%\\ Superior: NA\\ Limites de Controle:Inferior: 100\%\\ Superior: NA\end{tabular}                             \\ \hline
	\end{tabular}
	\end{table}

\subsection{Esforço}

	
	\begin{table}[H]
	\centering
	\caption{Esforço da Equipe}
	\label{my-label}
	\begin{tabular}{|l|l|}
	\hline
	Objetivo da medição & Identificar o esforço da equipe                                                                                                                                                                                            \\ \hline
	Formula             & $\sum$Horas trabalhadas                                                                                                                                                                                                    \\ \hline
	Escala da medição   & Racional                                                                                                                                                                                                                   \\ \hline
	Coleta              & \begin{tabular}[c]{@{}l@{}}Responsável: Rafael Akiyoshi\\ Peridiocidade: Semanalmente\\ Procedimentos: Entrevista\end{tabular}                                                                                             \\ \hline
	Análise             & \begin{tabular}[c]{@{}l@{}}Responsável: Gustavo Sabino\\ Procedimentos: Levando em consideração \\ as atividades realizadas da semana, esta \\ servirá de insumo para responder à \\ produtividade da equipe.\end{tabular} \\ \hline
	Meta                & \begin{tabular}[c]{@{}l@{}}Limites de Especificação:\\ Inferior: 90\%\\ Superior: NA\\ Limites de Controle:Inferior: 70\%\\ Superior: NA\end{tabular}                                                                      \\ \hline
	\end{tabular}
	\end{table}

	
	\begin{table}[H]
	\centering
	\caption{Esforço dos monitores}
	\label{my-label}
	\begin{tabular}{|l|l|}
	\hline
	Objetivo da medição & Identificar o esforço dos monitores                                                                                                                                                                                            \\ \hline
	Formula             & $\sum$Horas trabalhadas                                                                                                                                                                                                    \\ \hline
	Escala da medição   & Racional                                                                                                                                                                                                                   \\ \hline
	Coleta              & \begin{tabular}[c]{@{}l@{}}Responsável: Rafael Akiyoshi\\ Peridiocidade: Semanalmente\\ Procedimentos: Entrevista\end{tabular}                                                                                             \\ \hline
	Análise             & \begin{tabular}[c]{@{}l@{}}Responsável: Gustavo Sabino\\ Procedimentos: Levando em consideração \\ as atividades realizadas da semana, esta \\ servirá de insumo para responder à \\ produtividade da equipe.\end{tabular} \\ \hline
	Meta                & \begin{tabular}[c]{@{}l@{}}Limites de Especificação:\\ Inferior: 90\%\\ Superior: NA\\ Limites de Controle:Inferior: 70\%\\ Superior: NA\end{tabular}                                                                      \\ \hline
	\end{tabular}
	\end{table}

\subsection{Custo}


	\begin{table}[H]
	\centering
	\caption{Custo das atividades}
	\label{my-label}
	\begin{tabular}{|l|l|}
	\hline
	Objetivo da medição & Identificar o custo das atividades                                                                                                                                 \\ \hline
	Formula             & (Horas da atividade * Custo hora * Responsáveis)                                                                                                                   \\ \hline
	Escala da medição   & Racional                                                                                                                                                           \\ \hline
	Coleta              & \begin{tabular}[c]{@{}l@{}}Responsável: Gustavo Sabino\\ Peridiocidade: Semanalmente\\ Procedimentos: Entrevista\end{tabular}                                      \\ \hline
	Análise             & \begin{tabular}[c]{@{}l@{}}Responsável: Lucas Severo\\ Procedimentos: Dado o custo das atividades, \\ será levantado uma média de preço por atividade\end{tabular} \\ \hline
	Meta                & \begin{tabular}[c]{@{}l@{}}Limites de Especificação:\\ Inferior: 90\%\\ Superior: NA\\ Limites de Controle:Inferior: 70\%\\ Superior: NA\end{tabular}              \\ \hline
	\end{tabular}
	\end{table}

	\begin{table}[H]
	\centering
	\caption{Custo de adição de atividades de modelo de maturidade}
	\label{my-label}
	\begin{tabular}{|l|l|}
	\hline
	Objetivo da medição & \begin{tabular}[c]{@{}l@{}}Identificar o custo de adicionar uma \\ atividade do modelo de maturidade\end{tabular}                                                  \\ \hline
	Formula             & (Horas da atividade * Custo hora * Responsáveis)                                                                                                                   \\ \hline
	Escala da medição   & Racional                                                                                                                                                           \\ \hline
	Coleta              & \begin{tabular}[c]{@{}l@{}}Responsável: Gustavo Sabino\\ Peridiocidade: Semanalmente\\ Procedimentos: Entrevista\end{tabular}                                      \\ \hline
	Análise             & \begin{tabular}[c]{@{}l@{}}Responsável: Lucas Severo\\ Procedimentos: Dado o custo das atividades, \\ será levantado uma média de preço por atividade\end{tabular} \\ \hline
	Meta                & \begin{tabular}[c]{@{}l@{}}Limites de Especificação:\\ Inferior: 90\%\\ Superior: NA\\ Limites de Controle:Inferior: 70\%\\ Superior: NA\end{tabular}              \\ \hline
	\end{tabular}
	\end{table}



\section{Indicadores}


\subsection{Satisfação do cliente quanto à qualidade dos requisitos}
	Este será utilizado visando comparar os resultados com as medições de esforço e tamanho, e tem-se:
	\begin{itemize}
		\item Bom
		\item Médio
		\item Ruim
	\end{itemize}
\subsection{Satisfação do cliente quanto ao processo}
	Este será utilizado visando comparar os resultados com as medições de esforço e tamanho, e tem-se:
	\begin{itemize}
		\item Bom
		\item Médio
		\item Ruim
	\end{itemize}
\subsection{Satisfação do cliente quanto à qualidade da equipe}
	Este será utilizado visando comparar os resultados com as medições de esforço e tamanho, e tem-se:
	\begin{itemize}
		\item Bom
		\item Médio
		\item Ruim
	\end{itemize}

Por fim, teremos o indicador final do plano de medição, o qual será respondido ao final de todas as análises de dados:

\subsection{Qualidade do projeto com adesão à um modelo de maturidade}
	Este será utilizado visando a conclusão final do trabalho, e tem-se:
	\begin{itemize}
		\item Bom
		\item Médio
		\item Ruim
		\item Péssimo
	\end{itemize}