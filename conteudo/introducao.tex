\chapter{Introdução}

Com base no GQM (Goal, Question, Metric) que é um padrão e modelo de medição baseado em questionamentos,e no artigo ~\cite{borges2003modelo}, definimos as seções desse documento. O GQM define um modelo de medição que segue em 3 níveis: Objetivos, pergunts e por fim, métricas. Os seus passos ajudam a levantar esses níveis de medições em etapas, para facilitar o levantamento de métricas e indicadores.

\section{Contexto}

Este documento é a formulação inicial de um processo de medição para analisar a qualidade dos requisitos gerados no contexto da disciplina de Engenharia de Requisitos ministrada na Universidade de Brasília - Campus Gama. 
Na disciplina em questão é realizada a construção de um processo de engenharia de requisitos o qual envolve todos as atividades principais da Engenharia de Requisitos e a adesão de algumas atividades dos modelos de maturidade. A aderência aos modelos de maturidade pode ser benéfica se for adequada corretamente ao processo elaborado. 
Como resultado é esperado que os requisitos construídos possuam mais qualidade, pois foram seguidas atividades que têm como objetivo garantir o sucesso no processo de elicitação.
Sob essa perspectiva, é interessante buscar medidas que relacionem a qualidade dos requisitos elicitados pela equipe com a aderência às atividades dos modelos de maturidade.

\section{Formulação do problema}

O processo de engenharia de requisitos é composto por 5 atividades principais Elicitação, Análise e negociação, Documentação e Verificação e Validação e Gerência de requisitos, e todas essas atividades possuem tópicos dentro dos modelos de maturidade, seja CMMI ou MPSBr, e dentro destes, possuem diversos níveis de aprofundamento.
	
Entretanto, o processo de requisitos pode não conter todas as atividades estabelecidas nos modelos de maturidade por questões diversas como: custos, prazos para construção do produto de software e maturidade da equipe. 

Com base nessa premissa, podemos definir objetivos de medição e elaborar questões a serem respondidas com base nas análises das métricas coletadas ao longo do processo de medição. As respostas dessas questões darão suporte para que a equipe de medição relacione o processo de requisitos com os modelos de maturidade.

Este relacionamento dará suporte à todo o trabalho, ou seja, o quão aderido está o modelo de maturidade ao processo da equipe responsável pelo desenvolvimento do processo de engenharia de requisitos.
 
