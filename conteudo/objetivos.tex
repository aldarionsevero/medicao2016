\chapter{Objetivos}

Essa seção de Objetivos é analoga à etapa de Objetivos de Medição da fase de Definição do GQM.

Este trabalho visa encontrar uma relação entre a qualidade dos requisitos elicitados pela equipe responsável e a qualidade do processo adotado. É importante observar pontos essenciais para medidas de qualidade que atendam algumas das métricas fundamentais.
A escolha das métricas se baseiam na necessidade de informação para analise, principalmente, os custos de operação do processo de elicitação e a satisfação do cliente com relação ao produto final do processo. 
Os problemas são identificados como Temas para que no futuro possam ser traduzidas em objetos mensuráveis. As tabelas a seguir mostram os Temas que foram identificados.  

\begin{table}[H]
\centering
\caption{Tema 1}
\label{my-label}
\begin{tabular}{|l|l|}
\hline
Analisar              & Satisfação do cliente                          \\ \hline
Com o Proposito de    & Melhorar o entendimento dos requisitos gerados \\ \hline
Com respeito a        & Expectativas e relacionamento com o grupo      \\ \hline
Sob ponto de Vista de & Cliente                                        \\ \hline
No contexto de        & Disciplina de Engenharia de Requisitos         \\ \hline
\end{tabular}
\end{table}

\begin{table}[H]
\centering
\caption{Tema 2}
\label{my-label}
\begin{tabular}{|l|l|}
\hline
Analisar              & Aderência aos modelos de maturidade            		\\ \hline
Com o Proposito de    & Melhorar o processo de Engenharia de Requisitos 	\\ \hline
Com respeito a        & Qualidade dos requisitos produzidos      			\\ \hline
Sob ponto de Vista de & Gerente da equipe                                   \\ \hline
No contexto de        & Disciplina de Engenharia de Requisitos       		\\ \hline
\end{tabular}
\end{table}

\begin{table}[H]
\centering
\caption{Tema 3}
\label{my-label}
\begin{tabular}{|l|l|}
\hline
Analisar              & O tempo gasto para completar as atividades do processo  \\ \hline
Com o Proposito de    & Melhorar a produtividade da equipe 						\\ \hline
Com respeito a        & Tempo dedicado em cada fase     						\\ \hline
Sob ponto de Vista de & Equipe de Engenharia de Requisitos                      \\ \hline
No contexto de        & Disciplina de Engenharia de Requisitos         			\\ \hline
\end{tabular}
\end{table}

\begin{table}[H]
\centering
\caption{Tema 3}
\label{my-label}
\begin{tabular}{|l|l|}
\hline
Analisar              & O custo das atividades  \\ \hline
Com o Proposito de    & Prever o custo do projeto						\\ \hline
Com respeito a        & Validação do projeto    						\\ \hline
Sob ponto de Vista de & Equipe de Engenharia de Requisitos                      \\ \hline
No contexto de        & Disciplina de Engenharia de Requisitos         			\\ \hline
\end{tabular}
\end{table}

